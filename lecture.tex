% Options for packages loaded elsewhere
% Options for packages loaded elsewhere
\PassOptionsToPackage{unicode}{hyperref}
\PassOptionsToPackage{hyphens}{url}
\PassOptionsToPackage{dvipsnames,svgnames,x11names}{xcolor}
%
\documentclass[
  letterpaper,
  DIV=11,
  numbers=noendperiod]{scrartcl}
\usepackage{xcolor}
\usepackage{amsmath,amssymb}
\setcounter{secnumdepth}{-\maxdimen} % remove section numbering
\usepackage{iftex}
\ifPDFTeX
  \usepackage[T1]{fontenc}
  \usepackage[utf8]{inputenc}
  \usepackage{textcomp} % provide euro and other symbols
\else % if luatex or xetex
  \usepackage{unicode-math} % this also loads fontspec
  \defaultfontfeatures{Scale=MatchLowercase}
  \defaultfontfeatures[\rmfamily]{Ligatures=TeX,Scale=1}
\fi
\usepackage{lmodern}
\ifPDFTeX\else
  % xetex/luatex font selection
\fi
% Use upquote if available, for straight quotes in verbatim environments
\IfFileExists{upquote.sty}{\usepackage{upquote}}{}
\IfFileExists{microtype.sty}{% use microtype if available
  \usepackage[]{microtype}
  \UseMicrotypeSet[protrusion]{basicmath} % disable protrusion for tt fonts
}{}
\makeatletter
\@ifundefined{KOMAClassName}{% if non-KOMA class
  \IfFileExists{parskip.sty}{%
    \usepackage{parskip}
  }{% else
    \setlength{\parindent}{0pt}
    \setlength{\parskip}{6pt plus 2pt minus 1pt}}
}{% if KOMA class
  \KOMAoptions{parskip=half}}
\makeatother
% Make \paragraph and \subparagraph free-standing
\makeatletter
\ifx\paragraph\undefined\else
  \let\oldparagraph\paragraph
  \renewcommand{\paragraph}{
    \@ifstar
      \xxxParagraphStar
      \xxxParagraphNoStar
  }
  \newcommand{\xxxParagraphStar}[1]{\oldparagraph*{#1}\mbox{}}
  \newcommand{\xxxParagraphNoStar}[1]{\oldparagraph{#1}\mbox{}}
\fi
\ifx\subparagraph\undefined\else
  \let\oldsubparagraph\subparagraph
  \renewcommand{\subparagraph}{
    \@ifstar
      \xxxSubParagraphStar
      \xxxSubParagraphNoStar
  }
  \newcommand{\xxxSubParagraphStar}[1]{\oldsubparagraph*{#1}\mbox{}}
  \newcommand{\xxxSubParagraphNoStar}[1]{\oldsubparagraph{#1}\mbox{}}
\fi
\makeatother


\usepackage{longtable,booktabs,array}
\usepackage{calc} % for calculating minipage widths
% Correct order of tables after \paragraph or \subparagraph
\usepackage{etoolbox}
\makeatletter
\patchcmd\longtable{\par}{\if@noskipsec\mbox{}\fi\par}{}{}
\makeatother
% Allow footnotes in longtable head/foot
\IfFileExists{footnotehyper.sty}{\usepackage{footnotehyper}}{\usepackage{footnote}}
\makesavenoteenv{longtable}
\usepackage{graphicx}
\makeatletter
\newsavebox\pandoc@box
\newcommand*\pandocbounded[1]{% scales image to fit in text height/width
  \sbox\pandoc@box{#1}%
  \Gscale@div\@tempa{\textheight}{\dimexpr\ht\pandoc@box+\dp\pandoc@box\relax}%
  \Gscale@div\@tempb{\linewidth}{\wd\pandoc@box}%
  \ifdim\@tempb\p@<\@tempa\p@\let\@tempa\@tempb\fi% select the smaller of both
  \ifdim\@tempa\p@<\p@\scalebox{\@tempa}{\usebox\pandoc@box}%
  \else\usebox{\pandoc@box}%
  \fi%
}
% Set default figure placement to htbp
\def\fps@figure{htbp}
\makeatother
\usepackage{svg}





\setlength{\emergencystretch}{3em} % prevent overfull lines

\providecommand{\tightlist}{%
  \setlength{\itemsep}{0pt}\setlength{\parskip}{0pt}}



 


\KOMAoption{captions}{tableheading}
\makeatletter
\@ifpackageloaded{tcolorbox}{}{\usepackage[skins,breakable]{tcolorbox}}
\@ifpackageloaded{fontawesome5}{}{\usepackage{fontawesome5}}
\definecolor{quarto-callout-color}{HTML}{909090}
\definecolor{quarto-callout-note-color}{HTML}{0758E5}
\definecolor{quarto-callout-important-color}{HTML}{CC1914}
\definecolor{quarto-callout-warning-color}{HTML}{EB9113}
\definecolor{quarto-callout-tip-color}{HTML}{00A047}
\definecolor{quarto-callout-caution-color}{HTML}{FC5300}
\definecolor{quarto-callout-color-frame}{HTML}{acacac}
\definecolor{quarto-callout-note-color-frame}{HTML}{4582ec}
\definecolor{quarto-callout-important-color-frame}{HTML}{d9534f}
\definecolor{quarto-callout-warning-color-frame}{HTML}{f0ad4e}
\definecolor{quarto-callout-tip-color-frame}{HTML}{02b875}
\definecolor{quarto-callout-caution-color-frame}{HTML}{fd7e14}
\makeatother
\makeatletter
\@ifpackageloaded{caption}{}{\usepackage{caption}}
\AtBeginDocument{%
\ifdefined\contentsname
  \renewcommand*\contentsname{Table of contents}
\else
  \newcommand\contentsname{Table of contents}
\fi
\ifdefined\listfigurename
  \renewcommand*\listfigurename{List of Figures}
\else
  \newcommand\listfigurename{List of Figures}
\fi
\ifdefined\listtablename
  \renewcommand*\listtablename{List of Tables}
\else
  \newcommand\listtablename{List of Tables}
\fi
\ifdefined\figurename
  \renewcommand*\figurename{Figure}
\else
  \newcommand\figurename{Figure}
\fi
\ifdefined\tablename
  \renewcommand*\tablename{Table}
\else
  \newcommand\tablename{Table}
\fi
}
\@ifpackageloaded{float}{}{\usepackage{float}}
\floatstyle{ruled}
\@ifundefined{c@chapter}{\newfloat{codelisting}{h}{lop}}{\newfloat{codelisting}{h}{lop}[chapter]}
\floatname{codelisting}{Listing}
\newcommand*\listoflistings{\listof{codelisting}{List of Listings}}
\makeatother
\makeatletter
\makeatother
\makeatletter
\@ifpackageloaded{caption}{}{\usepackage{caption}}
\@ifpackageloaded{subcaption}{}{\usepackage{subcaption}}
\makeatother
\usepackage{bookmark}
\IfFileExists{xurl.sty}{\usepackage{xurl}}{} % add URL line breaks if available
\urlstyle{same}
\hypersetup{
  pdftitle={Data Science, AI, and ML in Systems Engineering},
  pdfauthor={Aykut C. Satici},
  colorlinks=true,
  linkcolor={blue},
  filecolor={Maroon},
  citecolor={Blue},
  urlcolor={Blue},
  pdfcreator={LaTeX via pandoc}}


\title{Data Science, AI, and ML in Systems Engineering}
\usepackage{etoolbox}
\makeatletter
\providecommand{\subtitle}[1]{% add subtitle to \maketitle
  \apptocmd{\@title}{\par {\large #1 \par}}{}{}
}
\makeatother
\subtitle{From Classical Control to Intelligent Systems}
\author{Aykut C. Satici}
\date{2026-02-09}
\begin{document}
\maketitle


\subsection{Who am I?}\label{who-am-i}

\subsubsection{My Journey}\label{my-journey}

\begin{itemize}
\tightlist
\item
  \textbf{Education}: BS/MS in Mechatronics (Turkey), PhD in EE \& MS in
  Math (UTD).
\item
  \textbf{Research Training}: Postdoc at U. Naples (Italy) \& MIT (USA).
\item
  \textbf{Professorate}: Assistant \& Associate Professor at
  \textbf{Boise State University} (2017-2025).
\item
  \textbf{Currently}: Back home at \textbf{UT Dallas} as Associate
  Professor of Systems Engineering.
\end{itemize}

\begin{center}
\includegraphics[width=\linewidth,height=2.39583in,keepaspectratio]{images/sabanci_team.jpg}
\end{center}
\\
\begin{center}
\includegraphics[width=\linewidth,height=2.39583in,keepaspectratio]{images/spong-lab-650-2013-08.jpg}
\end{center}

\begin{itemize}
\tightlist
\item
  I've been around! I started in Turkey, did my PhD here at UTD, and
  then spent time in Italy and MIT.
\item
  I spent the last 9 years at Boise State where I built my research lab
  and received tenure.
\item
  I'm thrilled to be back at UTD starting this semester!
\end{itemize}

\subsection{BSU: Locomotion \& Contact}\label{bsu-locomotion-contact}

\begin{itemize}
\tightlist
\item
  \textbf{The Problem}: Robots struggle with uneven terrain and physical
  collisions.
\item
  \textbf{The Solutions}:

  \begin{itemize}
  \tightlist
  \item
    \textbf{Rimless Wheel}: Stability of passive walking.
  \item
    \textbf{Astria}: Drone/power-line interaction.
  \item
    \textbf{IWP}: Energy-shaping control for acrobatics.
  \end{itemize}
\end{itemize}

\begin{figure}

\begin{minipage}{0.50\linewidth}
\url{videos/iwp.mp4}\end{minipage}%
%
\begin{minipage}{0.50\linewidth}
\url{videos/astria.mp4}\end{minipage}%

\end{figure}%

\includegraphics[width=0.5\linewidth,height=\textheight,keepaspectratio]{images/rimless_wheel.webp}

\begin{itemize}
\tightlist
\item
  At BSU, I worked on the challenge of ``contact.'' Most robots are
  afraid of touching things too hard.
\item
  We used Systems Engineering to design controllers that handle these
  impacts gracefully.
\end{itemize}

\subsection{Systems Engineering in
Action}\label{systems-engineering-in-action}

\textbf{Drones}

\url{videos/quadrotor_lissajous.mp4}

\textbf{Soft Juggling}

\url{videos/soft_juggler.mp4}

\textbf{Manipulation}

\url{videos/pizza_tossing.mp4}

\begin{tcolorbox}[enhanced jigsaw, leftrule=.75mm, title=\textcolor{quarto-callout-note-color}{\faInfo}\hspace{0.5em}{Note}, colbacktitle=quarto-callout-note-color!10!white, titlerule=0mm, colback=white, toptitle=1mm, breakable, colframe=quarto-callout-note-color-frame, arc=.35mm, bottomrule=.15mm, rightrule=.15mm, bottomtitle=1mm, opacitybacktitle=0.6, left=2mm, coltitle=black, toprule=.15mm, opacityback=0]

Systems Engineering connects \textbf{Control Theory}, \textbf{Embedded
Systems}, and \textbf{Mechanical Design} to make these robots work in
the real world.

\end{tcolorbox}

\begin{itemize}
\tightlist
\item
  These videos show systems level integration.
\item
  Drones: Hardware + Real-time control.
\item
  Soft Robots: Complex modeling + Sensing.
\item
  Manipulation: High-speed coordination.
\end{itemize}

\subsection{Agenda}\label{agenda}

\begin{enumerate}
\def\labelenumi{\arabic{enumi}.}
\tightlist
\item
  \textbf{The Evolution}

  \begin{itemize}
  \tightlist
  \item
    From Clockwork to Neural Nets.
  \end{itemize}
\item
  \textbf{The Concepts}

  \begin{itemize}
  \tightlist
  \item
    DS vs.~AI vs.~ML.
  \end{itemize}
\item
  \textbf{The Lifecycle}

  \begin{itemize}
  \tightlist
  \item
    Where AI fits in the V-Model.
  \end{itemize}
\end{enumerate}

\begin{enumerate}
\def\labelenumi{\arabic{enumi}.}
\setcounter{enumi}{3}
\tightlist
\item
  \textbf{Live Case Study}

  \begin{itemize}
  \tightlist
  \item
    The Furuta Pendulum.
  \end{itemize}
\item
  \textbf{The Future}

  \begin{itemize}
  \tightlist
  \item
    Black boxes \& Ethics.
  \end{itemize}
\end{enumerate}

\section{Part 1: The Evolution}\label{part-1-the-evolution}

\subsection{Classical vs.~Intelligent
Systems}\label{classical-vs.-intelligent-systems}

\textbf{Deterministic}
\includegraphics[width=\linewidth,height=3.125in,keepaspectratio]{lecture_files/mediabag/Deterministic_proces.png}
\emph{Predictable. Static.}

\textbf{Stochastic}
\includegraphics[width=\linewidth,height=3.125in,keepaspectratio]{lecture_files/mediabag/Stochastic_process.png}
\emph{Adaptive. Learning.}

\textbf{Script:} ``Historically, SE was about building clockwork. If I
turn gear A, gear B moves. Today, we build systems like the Mars Rover.
It encounters unknown terrain. We can't pre-program every reaction. It
has to learn.''

\subsection{Why We Need AI}\label{why-we-need-ai}

\includegraphics[width=\linewidth,height=4.16667in,keepaspectratio]{lecture_files/mediabag/Muxviz_GlobalRisk.png}

\subsubsection{The Complexity Gap}\label{the-complexity-gap}

\begin{itemize}
\tightlist
\item
  \textbf{Data Volume}: Terabytes/hour.
\item
  \textbf{Dimensionality}: Systems with 1000+ variables.
\item
  \textbf{Human Limits}: We cannot manually optimize these anymore.
\end{itemize}

\textbf{Script:} ``Why can't we just write `if-then' statements? Because
modern systems look like this (point to network map). They are too
complex for manual analysis.''

\section{Part 2: The Engineer's
Definition}\label{part-2-the-engineers-definition}

\subsection{Disentangling the
Buzzwords}\label{disentangling-the-buzzwords}

\includesvg[width=\linewidth,height=4.6875in,keepaspectratio]{lecture_files/mediabag/AI_hierarchy.svg}

\textbf{Script:} ``Let's be precise. \textbf{AI} is the machine acting
smart. \textbf{ML} is the algorithm learning from data. \textbf{Deep
Learning} is the neural network approach. \textbf{Data Science} is the
analysis of the logs.''

\subsection{The Engineering View}\label{the-engineering-view}

\subsubsection{Classical}\label{classical}

\[ y = f(x) \] \emph{We know the physics (F=ma).} \emph{We write the
equation.}

\subsubsection{Machine Learning}\label{machine-learning}

\[ y \approx \hat{f}(x) \] \emph{We have the data points.} \emph{The
computer finds the curve.}

\includegraphics[width=\linewidth,height=3.125in,keepaspectratio]{lecture_files/mediabag/640px-Linear_regress.png}

\textbf{Script:} ``To an engineer, ML is just `Function Approximation.'
We have a bunch of dots (data), and we ask the computer to draw the best
line through them. That line becomes our law.''

\section{Part 3: The Lifecycle}\label{part-3-the-lifecycle}

\subsection{The V-Model and AI}\label{the-v-model-and-ai}

\includegraphics[width=\linewidth,height=5.20833in,keepaspectratio]{lecture_files/mediabag/Systems_Engineering_.jpg}

\textbf{Script:} ``This is the Holy Grail of SE: The V-Model. Think of
it as the `Map of Engineering.'

\begin{itemize}
\tightlist
\item
  \textbf{The Basic V:} We go \textbf{down} into the details (Left) and
  then back \textbf{up} into the real world (Right).
\item
  \textbf{The AI Connection (Dual-Track):} AI is just a subsystem. The
  V-Model ensures the `Smart' AI code actually integrates with the
  `Real' hardware.
\item
  \textbf{The Requirement Shift:} Instead of writing rules (If-Then), we
  design the \textbf{Data}. We are engineering the environment the AI
  learns from.
\item
  \textbf{The Verification Gap:} We can't test every line of code in a
  neural network. Verification shifts from `Does it match logic?' to
  `Does it handle the statistical edge cases?'\,''
\end{itemize}

\subsection{Operations: The Digital
Twin}\label{operations-the-digital-twin}

\includegraphics[width=\linewidth,height=3.64583in,keepaspectratio]{lecture_files/mediabag/General_Electric_GEn.jpg}

\textbf{Physical Asset}

\includesvg[width=\linewidth,height=3.64583in,keepaspectratio]{lecture_files/mediabag/Turbofan_operation.svg}

\textbf{Digital Replica}

\textbf{Script:} ``We build a Digital Twin. The physical engine sends
data to the digital one. The digital one uses ML to predict failures
before they happen.''

\subsection{Astria: Digital Twin}\label{astria-digital-twin}

\begin{itemize}
\tightlist
\item
  \textbf{The Physical Asset}: Autonomous drone with a gripper.
\item
  \textbf{The Digital Replica}: High-fidelity physics-based simulation.
\item
  \textbf{Systems Goal}: Testing contact physics and control logic
  virtually before risking expensive hardware on power lines.
\end{itemize}

\begin{figure}

\begin{minipage}{0.50\linewidth}
\url{videos/astria.mp4}\end{minipage}%
%
\begin{minipage}{0.50\linewidth}
\url{videos/unclamped_then_clamped.mp4}\end{minipage}%

\end{figure}%

\textbf{Real-world (Left) vs.~Sim2Real Replica (Right)}

\textbf{Script:} ``Here is a concrete example from our research. We have
the physical Astria drone, and we have its high-fidelity digital
replica. Before we fly near high-voltage lines, we test every control
algorithm in the digital twin to ensure the system behaves as
expected.''

\subsection{Rimless Wheel: Robustness}\label{rimless-wheel-robustness}

\begin{itemize}
\tightlist
\item
  \textbf{Systems Goal}: Walking that doesn't fall when the ground
  changes.
\item
  \textbf{Non-examples}:

  \begin{itemize}
  \tightlist
  \item
    Falling over (unstable).
  \item
    Tripping on uneven ground (fragile).
  \end{itemize}
\item
  \textbf{Examples}:

  \begin{itemize}
  \tightlist
  \item
    Steady rhythm on flat ground.
  \item
    \textbf{Robustness}: Recovering from uneven terrain.
  \end{itemize}
\end{itemize}

\textbf{Non-examples (Not Working)}

\begin{figure}

\begin{minipage}{0.50\linewidth}
\url{videos/rw_even.mp4}\end{minipage}%
%
\begin{minipage}{0.50\linewidth}
\url{videos/rw_uneven_success.mp4}\end{minipage}%

\end{figure}%

\textbf{Examples (Working Well)}

\begin{figure}

\begin{minipage}{0.50\linewidth}
\url{videos/rw_failing.mp4}\end{minipage}%
%
\begin{minipage}{0.50\linewidth}
\url{videos/rw_uneven.mp4}\end{minipage}%

\end{figure}%

\section{Part 4: Live Case Study}\label{part-4-live-case-study}

\subsection{How AI Learns to Control}\label{how-ai-learns-to-control}

\includegraphics[width=\linewidth,height=3.64583in,keepaspectratio]{lecture_files/mediabag/640px-Reinforcement_.png}

\textbf{Reinforcement Learning}

\begin{enumerate}
\def\labelenumi{\arabic{enumi}.}
\tightlist
\item
  \textbf{Observe} state.
\item
  \textbf{Take Action} (Voltage).
\item
  \textbf{Get Reward} (+1 if upright).
\end{enumerate}

\textbf{Script:} ``We use Reinforcement Learning. The AI tries random
things. It fails. It learns. Eventually, it masters the physics.''

\subsection{Sim2Real: The Reality
Check}\label{sim2real-the-reality-check}

\textbf{Simulation}

\includesvg[width=\linewidth,height=3.125in,keepaspectratio]{lecture_files/mediabag/Reinforcement_learni.svg}

\emph{Perfect Math}

\textbf{Reality}

\includegraphics[width=\linewidth,height=3.125in,keepaspectratio]{lecture_files/mediabag/RotaryInvertedPendul.JPG}

\emph{Noise \& Friction}

\textbf{Script:} ``We train in simulation (perfect). We deploy to
reality (messy). If the model overfits, the robot crashes. We need
Robust Control.''

\subsection{Learning to Walk: RL in
Action}\label{learning-to-walk-rl-in-action}

\begin{itemize}
\tightlist
\item
  \textbf{The Goal}: Make the robot walk forward.
\item
  \textbf{The Process}:

  \begin{itemize}
  \tightlist
  \item
    Start with zero knowledge.
  \item
    Try random actions.
  \item
    Get rewarded for forward motion.
  \item
    \textbf{Result}: Emergent walking behavior.
  \end{itemize}
\end{itemize}

\url{videos/walking_RL.mp4}

\subsection{The Furuta Pendulum}\label{the-furuta-pendulum}

\begin{itemize}
\tightlist
\item
  \textbf{Goal}: Balance upright.
\item
  \textbf{Actuator}: Base Motor.
\item
  \textbf{Sensor}: Encoders.
\end{itemize}

\includegraphics[width=\linewidth,height=4.16667in,keepaspectratio]{images/furuta.jpg}

\textbf{Script:} (Walk to the table) ``Here it is. The Furuta Pendulum.
It has a motor, an arm, and a pendulum. It looks like a system. But is
it?''

\subsection{Integration: The Furuta
Pendulum}\label{integration-the-furuta-pendulum}

\textbf{Controller Architecture}
\includegraphics[width=\linewidth,height=4.16667in,keepaspectratio]{lecture_files/mediabag/640px-Artificial_neu.png}

\textbf{Performance (Pre-Move)} \url{videos/furuta.mp4}

\begin{itemize}
\tightlist
\item
  \textbf{Hardware}: Assembled (Physical).
\item
  \textbf{Controller}: Neural Network (Code).
\item
  \emph{Video shows successful operation from Boise State lab.}
\end{itemize}

\textbf{Script:} ``Here is the plan. On the left, we have the Neural
Network controller (the brain). On the right, we have the video of it
working perfectly in my previous lab. Currently, due to the move, the
hardware and software are disconnected. That is the \textbf{Integration
Gap} we need to solve.''

\section{Part 5: The Future}\label{part-5-the-future}

\subsection{The Black Box Problem}\label{the-black-box-problem}

\subsubsection{Classical Code}\label{classical-code}

\texttt{IF\ speed\ \textgreater{}\ 50\ THEN\ brake} \emph{(Traceable)}

\subsubsection{Neural Network}\label{neural-network}

\texttt{0.23\ *\ x1\ +\ 0.99\ *\ x2\ ...} \emph{(Opaque)}

\includegraphics[width=\linewidth,height=3.125in,keepaspectratio]{lecture_files/mediabag/640px-Colored_neural.png}

\textbf{Script:} ``Traditional code is readable. Neural networks are
millions of opaque numbers. Traceability is a massive open problem in
SE.''

\subsection{Ethics \& Responsibility}\label{ethics-responsibility}

\includesvg[width=\linewidth,height=4.16667in,keepaspectratio]{lecture_files/mediabag/Trolley_Problem.svg}

\textbf{Script:} ``If an autonomous car hits a pedestrian to save the
passenger, who is responsible? As we build autonomous systems, we are
engineering moral agents.''

\section{Conclusion}\label{conclusion}

\subsection{Summary}\label{summary}

\includesvg[width=\linewidth,height=2.08333in,keepaspectratio]{lecture_files/mediabag/Gears-solid.svg}

\begin{enumerate}
\def\labelenumi{\arabic{enumi}.}
\tightlist
\item
  \textbf{Systems are changing}: From static to dynamic.
\item
  \textbf{Tools}: Python/MATLAB + Control Theory.
\item
  \textbf{Mindset}: AI is just a subsystem. Integration is key.
\end{enumerate}

\subsection{Q\&A}\label{qa}

\textbf{Aykut Satici}\\
\emph{Department of Systems Engineering}\\
\emph{University of Texas at Dallas}

\emph{Questions?}




\end{document}
